\documentclass[UTF8,12pt]{ctexart}
\usepackage{ctex,amsmath,amsthm,amsfonts,amssymb,bm,mathrsfs}
\usepackage{graphicx,float,geometry,subfigure,booktabs}
\usepackage{hyperref,cleveref,multicol,multirow,lipsum,lmodern,tcolorbox,color,arydshln,ulem,extarrows}
\usepackage{indentfirst}
\title{\zihao{-1}热统及格手册}
\author{\zihao{4}\tt 参宿四星云}
\date{\zihao{-4}\kaishu \today}
\geometry{papersize={210mm,297mm}}
\geometry{left=2.2cm,right=2.1cm,top=2.6cm,bottom=2.4cm}
\usepackage{fancyhdr}
\pagestyle{fancy}
\fancyhf{}
\fancyhead[L]{\S\thesection\ \leftmark}
\fancyhead[R]{\tt 参宿四星云}
\fancyfoot[C]{\thepage}
\usepackage{listings}
    \tcbuselibrary{skins, breakable, theorems}
%用法:$\stressbox{Formula}$·
    \newcommand\stressbox{\tcboxmath[colframe=red,colback=yellow!50!white]}
    \newcommand\stressarea{\tcboxmath[colframe=yellow!50!white,colback=yellow!50!white]}
    \newcommand\stress{\tcboxmath[colframe=yellow!50!white,colback=yellow!50!white,left=0mm,right=0mm,top=0mm,bottom=0mm]}
    \tcbset{colback=white}

    \newenvironment{itemizer}{\begin{itemize}}{\end{itemize}}
    \tcolorboxenvironment{itemizer}{blanker,before skip=12pt,after skip=12pt,borderline west={3mm}{0pt}{red}}
    \newenvironment{itemizeg}{\begin{itemize}}{\end{itemize}}
    \tcolorboxenvironment{itemizeg}{blanker,before skip=12pt,after skip=12pt,borderline west={3mm}{0pt}{green!66!black}}
    \newenvironment{itemizeb}{\begin{itemize}}{\end{itemize}}
    \tcolorboxenvironment{itemizeb}{blanker,before skip=12pt,after skip=12pt,borderline west={3mm}{0pt}{blue}}
    
\usepackage{varwidth}
\usepackage{titlesec}
\usepackage{titletoc}
%标题
    \titleformat{\part}{\centering\Huge\bfseries}{第\,\chinese{part}\,部分}{1em}{}
    \titleformat{\section}{\Large\bfseries}{\arabic{section}.}{1em}{}
    \titleformat{\subsection}{\large\bfseries}{{\arabic{section}.\arabic{subsection}}}{1em}{}
    \titleformat{\subsubsection}{\bfseries}{{\arabic{section}.\arabic{subsection}.\arabic{subsubsection}}}{1em}{}
    \titlespacing*{\part}{0pt}{-20pt}{20pt}
    \titlespacing*{\section}{0pt}{10pt}{10pt}
    \titlespacing*{\subsection}{0pt}{6pt}{6pt}
    \titlespacing*{\subsubsection}{0pt}{4pt}{4pt}
    %\titleformat{\subsection}{\bf}{\arabic{section}.\arabic{subsection}}{1em}{}
%目录
    \renewcommand{\contentsname}{\vspace*{-1.65cm}}
    \ctexset{section={name={\S}}}
    \titlecontents{part}[0em]{\vspace*{0.3cm}\heiti \Large}{\contentslabel{0em}}{}{\titlerule*[0.68pc]{}}
    \titlecontents{section}[2.2em]{\vspace*{0.3cm}\bf \large}{\contentslabel{2.0em}}{}{\titlerule*[0.68pc]{$\cdot$}\contentspage}

\newcommand{\I}{{\rm i}}
\newcommand{\pa}{\partial}
\newcommand{\tred}{\textcolor{red}}
\newcommand{\tpur}{\textcolor[RGB]{180,0,255}}
\newcommand{\tblue}{\textcolor[RGB]{100,24,255}}
\newcommand{\tblu}{\textcolor[RGB]{100,24,255}}
\newcommand{\tye}{\textcolor[RGB]{210,180,0}}
\newcommand{\tgreen}{\textcolor{green!75!black}}
\newcommand{\tgre}{\textcolor{green!75!black}}
\newcommand{\tgra}{\textcolor{white!50!black}}
%\newcommand{\cm}{\textcolor{green!60!black!60!white}}
\newcommand{\hdash}{\hdashline[2pt/2.5pt]}
\newcommand{\til}{\textasciitilde}
\newcommand{\dbar}{{\; \bar{} \hspace{-0.3em} \mathrm d}}


\tikzset{coltria/.style={fill=red!15!white}}
\newtcolorbox{ebox}[1][]{empty,
breakable,height fixed for=first and middle,
leftrule=5mm,left=2mm,
frame style={fill,top color=red!10!yellow!80!black,bottom color=red!10!yellow!70!black,middle color=red!10!yellow!77!black},
colback=red!10!yellow!20!white,
% code for unbroken boxes:
frame code={\path[tcb fill frame] (frame.south west)--(frame.north west)
--([xshift=-5mm]frame.north east)--([yshift=-5mm]frame.north east)
--([yshift=5mm]frame.south east)--([xshift=-5mm]frame.south east)--cycle; },
interior code={\path[tcb fill interior] (interior.south west)--(interior.north west)
--([xshift=-4.8mm]interior.north east)--([yshift=-4.8mm]interior.north east)
--([yshift=4.8mm]interior.south east)--([xshift=-4.8mm]interior.south east)
--cycle; },
% code for the first part of a break sequence:
skin first is subskin of={emptyfirst}{%
frame code={\path[tcb fill frame] (frame.south west)--(frame.north west)
--([xshift=-5mm]frame.north east)--([yshift=-5mm]frame.north east)
--(frame.south east)--cycle;
\path[coltria] ([xshift=2.5mm,yshift=1mm]frame.south west) -- +(120:2mm)
-- +(60:2mm)-- cycle; },
interior code={\path[tcb fill interior] (interior.south west|-frame.south)
--(interior.north west)--([xshift=-4.8mm]interior.north east)
--([yshift=-4.8mm]interior.north east)--(interior.south east|-frame.south)
--cycle; },
},%
% code for the middle part of a break sequence:
skin middle is subskin of={emptymiddle}{%
frame code={\path[tcb fill frame] (frame.south west)--(frame.north west)
--(frame.north east)--(frame.south east)--cycle;
\path[coltria] ([xshift=2.5mm,yshift=-1mm]frame.north west) -- +(240:2mm)
-- +(300:2mm) -- cycle;
\path[coltria] ([xshift=2.5mm,yshift=1mm]frame.south west) -- +(120:2mm)
-- +(60:2mm) -- cycle;
},
interior code={\path[tcb fill interior] (interior.south west|-frame.south)
--(interior.north west|-frame.north)--(interior.north east|-frame.north)
--(interior.south east|-frame.south)--cycle; },
},
% code for the last part of a break sequence:
skin last is subskin of={emptylast}{%
frame code={\path[tcb fill frame] (frame.south west)--(frame.north west)
--(frame.north east)--([yshift=5mm]frame.south east)
--([xshift=-5mm]frame.south east)--cycle;
\path[coltria] ([xshift=2.5mm,yshift=-1mm]frame.north west) -- +(240:2mm)
-- +(300:2mm) -- cycle;
251
},
interior code={\path[tcb fill interior] (interior.south west)
--(interior.north west|-frame.north)--(interior.north east|-frame.north)
--([yshift=4.8mm]interior.south east)--([xshift=-4.8mm]interior.south east)
--cycle; },
},
#1}

\begin{document}
\everymath{\displaystyle}
\maketitle
%\thispagestyle{empty}

\begin{table}[H]
    \centering  \renewcommand\arraystretch{1.3}
    %\caption{}
    %\label{}
    %\resizebox{1.0\linewidth}{!}{
    %\setlength{\tabcolsep}{1mm}{
    \begin{tabular}{c|cc||ccccccccccccccccc}
    \hline
    系统的种类&物质交换&能量交换&系综的种类&宏观条件\\
    \hline
    孤立系统&$\times$&$\times$&微正则系综&$N\quad V\quad E$\\
    闭系    &$\times$&$\tred{\sqrt{}}$&正则系综&$N\quad V\quad \tred{T}$\\
    开系    &$\tblu{\sqrt{}}$&$\tred{\sqrt{}}$&巨正则系综&$\tblu{\mu}\quad V\quad \tred{T}$\\
    \hline
    \end{tabular}	%} %}
\end{table}

\paragraph*{数学技巧:}\tred{偏微分读取、隐函数求导(3个变量)、链式法则、}\tgra{雅克比行列式*、}\tred{$\Gamma$函数、高斯积分、}\tgra{积分因子*、拉格朗日乘子法*、}\tred{斯特林公式}……

\setcounter{tocdepth}{1}
\tableofcontents

\newpage
\section{闭系的基本推导}

\subsection{热力学第一定律}
$$\Delta U=Q+W=\tred{\text{吸热}}-\tblu{\text{做功}}\qquad {\rm d}U=\dbar Q+\dbar W$$

\subsection{\tblue{做功(机械操控)}}

准静态过程外界对系统做功:
$$\tblu{\dbar W=-p{\rm d}V}=-Y{\rm d}y=-\sigma{\rm d}A=-U{\rm d}q$$

\subsection{热力学第二定律}
$$\left(\Delta S\right)_{\text{绝热}}\geqslant 0$$

1) 热力学温标/理想气体温标$T$$\quad \longrightarrow \quad$
${\rm d}S\geqslant \frac{\dbar Q}{T}$

2) 直接定义广延态函数\quad $S(U,y)=S(U,V)\quad \longrightarrow \quad
    T:=\left(\frac{\pa U}{\pa S}\right)_V\geqslant 0$


\subsection{\tred{热量交换}}

准静态过程系统吸收热量:
$$\tred{\dbar Q=T{\rm d}S}$$

\paragraph*{状态方程:}
$$f(p,V,T)=0$$

\begin{table}[H]
    \centering  \renewcommand\arraystretch{1.8}
    \resizebox{1.0\linewidth}{!}{
    %\setlength{\tabcolsep}{1mm}{
    \begin{tabular}{rlllllcccccc}
    %\hline
    \textbf{热容}\\
    基本定义&$C=\frac{\dbar Q}{{\rm d}T}$\\
    等容热容&$C_V=\left(\frac{\dbar Q}{{\rm d}T}\right)_V$
        &$=T\left(\frac{\pa S}{\pa T}\right)_V$
        &$=\left(\frac{\pa U}{\pa T}\right)_V$
        &$\xlongequal[]{\text{理想气体}}\frac{{\rm d}U}{{\rm d}T}$\\
    等压热容&$C_p=\left(\frac{\dbar Q}{{\rm d}T}\right)_p$
        &$=T\left(\frac{\pa S}{\pa T}\right)_p
        =\left(\frac{\pa U}{\pa T}\right)_p+p\left(\frac{\pa V}{\pa T}\right)_p$
        &$=\left(\frac{\pa H}{\pa T}\right)_p$
        &$\xlongequal[]{\text{理想气体}}\frac{{\rm d}H}{{\rm d}T}$\\
    %\hline
    \end{tabular}	} %}
\end{table}

\begin{table}[H]
    \centering  \renewcommand\arraystretch{1.9}
    %\caption{}
    %\label{}
    %\resizebox{1.0\linewidth}{!}{
    %\setlength{\tabcolsep}{1mm}{
    \begin{tabular}{cccccccccccccccccccc}
    \hline
    导出量的定义    &导出量的微分关系                 &麦克斯韦关系&使其不变的可逆过程\\
    \hline
    $\Delta U=Q+W$  &${\rm d}U=\tred{+T{\rm d}S}\tblu{-p{\rm d}V}$ 
    &$\left(\frac{\pa\tred{T}}{\pa\tblu{V}}\right)_{\tred{S}}
    =-\left(\frac{\pa\tblu{p}}{\pa\tred{S}}\right)_{\tblu{V}}$
    &理想气体恒温过程\\
    $H=U+pV$        &${\rm d}H=\tred{+T{\rm d}S}\tblu{+V{\rm d}p}$ 
    &$\left(\frac{\pa\tred{T}}{\pa\tblu{p}}\right)_{\tred{S}}
    =+\left(\frac{\pa\tblu{V}}{\pa\tred{S}}\right)_{\tblu{p}}$
    &理想气体恒温过程\\
    $F=U-TS$        &${\rm d}F=\tred{-S{\rm d}T}\tblu{-p{\rm d}V}$
    &$\left(\frac{\pa\tred{S}}{\pa\tblu{V}}\right)_{\tred{T}}
    =+\left(\frac{\pa\tblu{p}}{\pa\tred{T}}\right)_{\tblu{V}}$
    &恒温恒容过程\\
    $G=F+pV$        &${\rm d}G=\tred{-S{\rm d}T}\tblu{+V{\rm d}p}$
    &$\left(\frac{\pa\tred{S}}{\pa\tblu{p}}\right)_{\tred{T}}
    =-\left(\frac{\pa\tblu{V}}{\pa\tred{T}}\right)_{\tblu{p}}$
    &恒温恒压过程\\
    \hline
    $S\overset{*}{=}\frac{\dbar Q}{T}$&&&绝热过程\\

    \hline
    \end{tabular}	%} %}
\end{table}

\begin{ebox}
    \tblue{例1. 求温度不变时焓随压强的变化率与物态方程的关系。(P45 Eq.2.2.10)}
    \begin{equation}\begin{aligned}\notag
        \left(\frac{\pa H}{\pa p}\right)_T
        &=\left(\frac{\tgre{T\pa S+V\pa p}}{\pa p}\right)_T\\
        &=T\left(\frac{\pa S}{\pa p}\right)_T+V\\
        &=-T\left(\frac{\pa V}{\pa T}\right)_p+V
    \end{aligned}\end{equation}

\end{ebox}

\newpage

\section{开系的基本推导}

\tye{广延量:}$\tred{S}\quad \tblue{V}\quad\!\tgre{N}\quad U\quad H\quad F\quad G$

强度量:$\tred{T}\quad \tblue{p}\quad \tgre{\mu}$

\begin{table}[H]
    \centering  \renewcommand\arraystretch{1.7}
    \resizebox{1.0\linewidth}{!}{
    \setlength{\tabcolsep}{5mm}{
    \begin{tabular}{cclccccccccccccccccc}
    \hline
    &${\rm d}U=\tred{+T{\rm d}S}\tblu{-p{\rm d}V}\tgre{+\mu{\rm d}N}$
    &$U(\tye{S},\tye{V},\tye{N})=N\cdot u(\tye{S}/N,\tye{V}/N)$
    \\
    &${\rm d}H=\tred{+T{\rm d}S}\tblu{+V{\rm d}p}\tgre{+\mu{\rm d}N}$ 
    &$H(\tye{S},p,\tye{N})=N\cdot h(\tye{S}/N,p)$
    \\
    &${\rm d}F=\tred{-S{\rm d}T}\tblu{-p{\rm d}V}\tgre{+\mu{\rm d}N}$
    &$F(T,\tye{V},\tye{N})=N\cdot f(T,\tye{V}/N)$
    \\
    &${\rm d}G=\tred{-S{\rm d}T}\tblu{+V{\rm d}p}\tgre{+\mu{\rm d}N}$
    &$G(T,p,\tye{N})=\underset{\text{成功分离变量}}{\uline{N\cdot g(p,T)}}=N\mu$
    \\\hdash
    $J=F-\mu N=-pV$
    &${\rm d}J=\tred{-S{\rm d}T}\tblu{-p{\rm d}V}\tgre{-N{\rm d}\mu}$
    &$J(T,\tye{V},\mu)$
    \\
    \hline
    \end{tabular}	} }
\end{table}

\subsection{平衡条件}

$$\delta S=0\quad \Longrightarrow\quad 
\begin{cases}
    \tred{T^\alpha=T^\beta}&\text{热平衡条件}\\
    \tblu{p^\alpha=p^\beta}&\text{力学平衡条件}\\
    \tgre{\mu^\alpha=\mu^\beta}&\text{相变平衡条件}\\
\end{cases}$$

\subsection{多元复相系的热力学}
等温等压下,平衡态时$G$最小(吉布斯判据):
$$\delta G=0$$
整个系统达到平衡时,对于两相中均存在的任一组元,这个组元在两相中的化学势相等:
$$\mu_i^\alpha=\mu_i^\beta\quad(i=1,\cdots,k)$$

\newpage
\begin{ebox}
\tblue{例2. (P110 习题4.4)}
\begin{figure}[H]
    \centering
    \includegraphics[width=0.99\linewidth]{fig/4.4.jpg}
\end{figure}
\paragraph*{(a)}相变平衡时,溶剂在液、气两相的化学势相等:
$$\mu_1=\mu_1'$$
溶剂在溶液中的摩尔分数为$1-x$,在蒸气中的摩尔分数为1,那么
$$g_1+RT\ln(1-x)=g_1'$$

\paragraph*{(b)}令$T$不变,上式对$p$求偏导:
$$\left(\frac{\pa g_1}{\pa p}\right)_T-\frac{RT}{1-x}\tblu{\left(\frac{\pa x}{\pa p}\right)_T}=\left(\frac{\pa g_1'}{\pa p}\right)_T$$
已知
$$g=-s{\rm d}T+v{\rm d}p$$
那么
$$\left(\frac{\pa g}{\pa p}\right)_T=v$$
那么
$$v_1-\frac{RT}{1-x}\tblu{\left(\frac{\pa x}{\pa p}\right)_T}=v_1'$$
忽略溶剂液相摩尔体积$v_1$,那么
$$-\frac{RT}{1-x}\tblu{\left(\frac{\pa x}{\pa p}\right)_T}=v_1'$$
假设蒸气是理想气体$pv_1'=RT$,那么
$$\tblu{\left(\frac{\pa x}{\pa p}\right)_T}=-\frac{1-x}{p}$$

\end{ebox}

\subsection{摩尔潜热}

定义摩尔潜热:
$$L=T(S_{\rm m}^\beta-S_{\rm m}^\alpha)$$
\tgre{注:只有一级相变才有摩尔潜热。}

\paragraph*{一些结论:}\
\begin{itemizeg}
    \item 克拉伯龙方程
    $$\frac{{\rm d}p}{{\rm d}T}=\frac{L}{T(V_{\rm m}^\beta-V_{\rm m}^\alpha)}$$
    
    \item 理想气体的蒸气压方程(忽略凝聚相体积)
    $$\ln p=-\frac{L}{RT}+A$$
    
\end{itemizeg}

\subsection{范氏气体相变、相变的分类(跳过)}

\subsection{吉布斯关系(跳过)}

\subsection{混合理想气体、吉布斯佯谬(跳过)}

道尔顿分压定律:混合理想气体情况下,
$$p=\sum_i p_i$$
$$p_i=n_i\frac{RT}{V}\quad\Longrightarrow\quad p=n\frac{RT}{V}$$

\subsection{单相化学平衡(跳过)}

等温等压,平衡态$G$最小(吉布斯判据):
$$\delta G=0\quad\Longrightarrow\quad \sum_i\nu_i\mu_i=0$$
定压平衡常量
$$K_p(T)=\prod_ip_i^{\nu_i}$$

\subsection{热力学第三定律:能斯特定理(跳过)}

\newpage
\section{微观态}

\subsection{“微观态”、等概率原理$\Longrightarrow$三种分布$\Longrightarrow$速率分布}

\begin{figure}[H]
%    \subfigure[]
    {
    \begin{minipage}{8.16cm}
    \centering
    \includegraphics[width=0.98\linewidth]{fig/wfunc.jpg}
    \end{minipage}\label{}
    }
%    \subfigure[]
    {
    \begin{minipage}{8.16cm}
    \centering
    \includegraphics[width=0.98\linewidth]{fig/mstate.jpg}
    \end{minipage}\label{}
    }
%    \caption{}
%    \label{}
\end{figure}

\begin{table}[H]
    \centering  \renewcommand\arraystretch{2.2}
    %\caption{}
    %\label{}
    \resizebox{1.0\linewidth}{!}{
    %\setlength{\tabcolsep}{1mm}{
    \begin{tabular}{ccp{2.4cm}cp{3.9cm}cccc}
    \hline
    分布&粒子相容性&粒子全同性&能级$\epsilon_i$上的\tblue{微观态数}&每个微观态的最概然粒子数\\
    \hline
    B.E.&可相容&不可分辨&$\frac{(\omega_i+a_i-1)!}{(\omega_i-1)!a_i!}$
    &$f_i=\frac{a_i^*}{\omega_i}=\frac{1}{e^{\alpha+\beta\epsilon_i}-1}$\\
    %\hdash
    F.D.&不相容&不可分辨&$\frac{\omega_i!}{a_i!(\omega_i-a_i)!}$
    &$f_i=\frac{a_l^*}{\omega_i}=\frac{1}{e^{\alpha+\beta\epsilon_i}+1}$\\
    \hdash
    \tgre{M.B.}&\tgre{可相容*}&\tgre{可分辨*(吉布斯修正)}&$\tgre{\frac{\omega_i^{a_i}}{a_i!}}$
    &$f_i=\frac{a_i^*}{\omega_i}=e^{-\alpha-\beta\epsilon_i}$\\
    \hline
\end{tabular}	} %}
\end{table}
\tgre{注:$\alpha=-\frac{\mu}{kT}$,$\beta=\frac{1}{kT}$。}

\subsubsection{用玻尔兹曼分布推导 $D$ 维点粒子理想气体的速率分布}

\tgre{(小建议:参考课本,为了避免乘起来的项太混乱,表达式全部写成无量纲的形式。)}

粒子的相空间中(限定在$D$维体积为$L^D$的位形空间),${\rm d}p_1\cdots{\rm d}p_D$中的微观态数为
$$\frac{{\rm d}\omega}{h^D}
=\frac{{\rm d}x_1\cdots{\rm d}x_D\ {\rm d}p_1\cdots{\rm d}p_D}{h^D}
=\tred{\frac{L^D}{h^D}{\rm d}p_1\cdots{\rm d}p_D}$$

玻尔兹曼分布:能量为$\epsilon_i$的每个微观态的最概然粒子数为
$$f_i=e^{-\alpha-\beta\epsilon_i}=\tblue{e^{-\alpha}\exp\left[-\frac{\beta}{2m}\left(p_1^2+\cdots+p_D^2\right)\right]}$$

二者相乘,${\rm d}p_1\cdots{\rm d}p_D$中的最概然粒子数为
$$\tblue{e^{-\alpha}\exp\left[-\frac{\beta}{2m}\left(p_1^2+\cdots+p_D^2\right)\right]}
\tred{\frac{L^D}{h^D}{\rm d}p_1\cdots{\rm d}p_D}$$

从$p_j=mv_j$,$\beta=\frac{1}{kT}$,${\rm d}v_1\cdots{\rm d}v_D$中的最概然粒子数为:
$$\tblue{e^{-\alpha}\exp\left[-\frac{m}{2kT}\left(v_1^2+\cdots+v_D^2\right)\right]}
\tred{\frac{L^Dm^D}{h^D}{\rm d}v_1\cdots{\rm d}v_D}$$

计算总粒子数(速度分布的归一化系数)
\begin{equation}\begin{aligned}\label{}\notag
    N&=\int_0^\infty\cdots\int_0^\infty
    \tblue{e^{-\alpha}\exp\left[-\frac{m}{2kT}\left(v_1^2+\cdots+v_D^2\right)\right]}
    \tred{\frac{L^Dm^D}{h^D}{\rm d}v_1\cdots{\rm d}v_D}\\
    \tgra{\underset{\text{($\Gamma$函数)}}{\small\text{使用高斯积分}}}
    &=\tblue{e^{-\alpha}}\tred{\frac{L^Dm^D}{h^D}}\cdot
    \left(\sqrt{\frac{2kT}{m}\pi}\right)^D\\
    &=e^{-\alpha}L^D\left(\frac{2\pi mkT}{h^2}\right)^{D/2}
\end{aligned}\end{equation}

玻尔兹曼速度分布$f(v_1,\cdots,v_D)$为
\begin{equation}\begin{aligned}\label{}\notag
    f(v_1,\cdots,v_D)\ \tpur{{\rm d}v_1\cdots{\rm d}v_D}&=\frac{1}{N}\cdot\tblue{e^{-\alpha}\exp\left[-\frac{m}{2kT}\left(v_1^2+\cdots+v_D^2\right)\right]}
    \tred{\frac{L^Dm^D}{h^D}{\rm d}v_1\cdots{\rm d}v_D}\\
    &=\left(\frac{m}{2\pi kT}\right)^{D/2}\exp\left[-\frac{m}{2kT}\left(v_1^2+\cdots+v_D^2\right)\right]\tpur{{\rm d}v_1\cdots{\rm d}v_D}
\end{aligned}\end{equation}

\begin{ebox}
\tblue{一些二、三级结论}

\

速率分布$\tred{f(v)}$为
\begin{equation}\begin{aligned}\label{}\notag
    \tred{f(v)}\ {\rm d}v&=\underset{\text{$D-1$个}}{\underbrace{\int\cdots\int}}\ f(v_1,\cdots,v_D)\ {\rm d}v_1\cdots{\rm d}v_D\\
    &=\underset{v^2=v_1^2+\cdots+v_D^2}{\uuline{f(v_1,\cdots,v_D)}}\cdot\left(D\text{维球表面积}\right)\cdot {\rm d}v
\end{aligned}\end{equation}

\tcbline

平均速率$\bar v$为
$$\bar v=\int_0^\infty\tred{f(v)}v{\rm d}v$$

方均根速率$v_{\rm rms}$为
$$v_{\rm rms}^2=\int_0^\infty\tred{f(v)}v^2{\rm d}v$$

最概然速率$v_p$为
$$\left.\frac{{\rm d}\tred{f(v)}}{{\rm d}v}\right|_{v=v_p}=0$$

\end{ebox}

\tgre{Tip:玻色爱因斯坦凝聚和费米气体来不及整理了,,,,,,,}





\newpage
\section{系综理论基础}

\subsection{相空间}

\begin{figure}[H]
    \subfigure[线性谐振⼦本征态和相空间的关系]
    {
    \begin{minipage}{6cm}
    \centering
    \includegraphics[width=0.97\linewidth]{fig/dot.jpg}
    \end{minipage}\label{}
    }
    %\subfigure[]
    {
    \begin{minipage}{10cm}
    \centering
    \includegraphics[width=0.97\linewidth]{fig/rt3-14.jpg}
    \end{minipage}\label{}
    }
    %\caption{}
    %\label{}
\end{figure}

课本P209 Eq(9.2.7):刘维尔定理$\quad\overset{\text{经典表达}}{\Longrightarrow}\quad$等概率原理(微正则分布)

\subsection{系综}

\begin{figure}[H]
    %\subfigure[]
    {
    \begin{minipage}{7cm}
    \centering
    \includegraphics[width=0.92\linewidth]{fig/rt1-7.jpg}
    \end{minipage}\label{}
    }
    %\subfigure[]
    {
    \begin{minipage}{9.3cm}
    \centering
    \includegraphics[width=0.97\linewidth]{fig/rt4-2.jpg}
    \end{minipage}\label{}
    }
    %\caption{}
    %\label{}
\end{figure}

\begin{figure}[H]
    \subfigure[微正则系综]
    {
    \begin{minipage}{5cm}
    \centering
    \includegraphics[width=0.97\linewidth]{fig/c1.jpg}
    \end{minipage}\label{}
    }
    \subfigure[正则系综]
    {
    \begin{minipage}{5cm}
    \centering
    \includegraphics[width=0.97\linewidth]{fig/c2.jpg}
    \end{minipage}\label{}
    }
    \subfigure[巨正则系综]
    {
    \begin{minipage}{5cm}
    \centering
    \includegraphics[width=0.97\linewidth]{fig/c3.jpg}
    \end{minipage}\label{}
    }
\end{figure}

\subsubsection*{三种常用系综}

\begin{table}[H]
    \centering  \renewcommand\arraystretch{1}
    \resizebox{1.0\linewidth}{!}{
    %\setlength{\tabcolsep}{1mm}{
    \begin{tabular}{|c|cc|cc|ccccccccccccccc}
    \hline\rule{0pt}{19pt}
    系综&微正则系综&\multicolumn{2}{|c|}{正则系综}&巨正则系综\\
    \hline\rule{0pt}{28pt}
    概率
    &$\frac{1}{\tred{\Omega(N,V,E)}}$
    &\multicolumn{2}{|c|}{
    $\frac{\exp(-\beta E_s)}{\tred{Z(N,V,\tblue{T})}}$
    }
    &$\frac{\exp(-\alpha N_r-\beta E_s)}{\tred{\Xi(\tgre{\mu},V,\tblue{T})}}$\\
    \hdash\rule{0pt}{65pt}
    配分函数
    &$\begin{aligned}
        &\quad\tred{\Omega(N,V,E)}\\
        &\tred{=\sum_{i\text{(简并态)}}1}\\
        &\ 
    \end{aligned}$
    &\multicolumn{2}{|c|}{$\begin{aligned}
        &\quad \tred{Z(N,V,\tblue{\beta})}\\
        &\ \tred{=\sum_{i\text{(态)}}e^{-\beta E_i}}\\
        &=\sum_{r\text{(能级)}}\Omega(N,V,E_r)\tblu{e^{-\beta E_r}}
    \end{aligned}$}
    &$\begin{aligned}
        &\quad \tred{\Xi(\tgre{\alpha},V,\tblue{\beta})}\\
        &\ \tred{=\sum_{s}\sum_{i\text{(态)}}e^{-\alpha N_s-\beta E_i}}\\
        &=\sum_{s}\sum_{r\text{(能级)}}\Omega(N_s,V,E_r)e^{-\alpha N_s-\beta E_r}\\
        &=\sum_{s}Z(N_s,V,\beta)\tgre{e^{-\alpha N_s}}
    \end{aligned}$\\
    \hdash
    $\begin{aligned}&\text{拉普拉斯}\\&\text{变换}\end{aligned}$
    &\multicolumn{2}{c}{
        $\ \mathscr{L}\left[\Omega(N,V,\tye{E})\right]=Z(N,V,\tye{\beta})\ $
    }
    &\multicolumn{2}{|c|}{
        $\mathscr{L}\left[Z(\tye{N},V,\beta)\right]=\Xi(\tye{\alpha},V,\beta)$
    }\\
    \hdash\rule{0pt}{37pt}
    熵展开式
    &$\begin{aligned}
        S&=k\ln\Omega\\
        &\
    \end{aligned}$
    &\multicolumn{2}{|c|}{
    $\begin{aligned}
        S&=k\left(\ln Z-\beta\frac{\pa\ln Z}{\pa\beta}\right)\\
        &=k\left(\ln Z+\beta\overline E\right)
    \end{aligned}$
    }
    &$\begin{aligned}
        S&=k\left(\ln\Xi-\alpha\frac{\pa\ln\Xi}{\pa\alpha}-\beta\frac{\pa\ln\Xi}{\pa\beta}\right)\\
        &=k\left(\ln\Xi+\alpha\overline N+\beta\overline E\right)
    \end{aligned}$\\
    \hdash
    特性函数\rule{0pt}{30pt}
    &
    &\multicolumn{2}{|c|}{
    $\begin{aligned}
        F&=-kT\ln Z\\
        {\rm d}F&=\tred{-S{\rm d}T}\tblu{-p{\rm d}V}
    \end{aligned}$
    }
    &$\begin{aligned}
        J&=-kT\ln\Xi\tgra{\ =-PV}\\
        {\rm d}J&=\tred{-S{\rm d}T}\tblu{-p{\rm d}V}\tgre{-N{\rm d}\mu}
    \end{aligned}$
    \\
    \hline
    \end{tabular}	} %}
\end{table}

\subsection{近独立粒子系统中的$Z$与$\Xi$}

\noindent 例:3维空间中的\ 单原子分子\ 经典(玻尔兹曼)\ 理想气体

$$E=\sum_{j=1\text{(粒子)}}^N\epsilon_j=\sum_{j=1}^{N}\frac{p_{j,x}^2+p_{j,y}^2+p_{j,z}^2}{2m}=\sum_{l=1}^{3N}\frac{p_l^2}{2m}$$
那么系统的(正则)配分函数为:
\begin{equation}\begin{aligned}\notag
    Z_N=Z(N,V,\beta)&=\tred{\sum_{i\text{(态)}}}e^{-\beta E_i}\\
    &=\tred{\int\text{…}\int \frac{{\rm d}^{3N}q{\rm d}^{3N}p}{N!h^{3N}}}e^{-\beta \tpur{E_i}}\\
    &=\int \frac{V^N}{N!h^{3N}}e^{-\beta E_i}{\rm d}^{3N}p\\
    &=\frac{V^N}{N!h^{3N}}\int\text{…}\int \exp\left(-\beta\ \tpur{\sum_{l=1}^{3N}\frac{p_l^2}{2m}}\right){\rm d}^{3N}p\\
    &=\frac{V^N}{N!h^{3N}}\int\text{…}\int\ \prod_{l=1}^{3N} \left[\exp\left(-\beta\frac{p_l^2}{2m}\right)\right]{\rm d}^{3N}p\\
    &=\frac{V^N}{N!h^{3N}}\left[\int\exp\left(-\beta\frac{p_l^2}{2m}\right){\rm d}p_l\right]^{3N}\\
    &=\frac{V^N}{N!h^{3N}}\left(\sqrt{\frac{2\pi m}{\beta}}\right)^{3N}\\
    &=\frac{1}{N!}\left[\frac{V}{h^3}\left(2\pi mkT\right)^{3/2}\right]^N
\end{aligned}\end{equation}

注意到,取$N=1$时,$Z=\frac{V}{h^3}\left(2\pi mkT\right)^{3/2}$,记为$Z_1$,那么有
$$Z_N=\frac{1}{N!}Z_1^N$$

系统的巨(正则)配分函数为:

\begin{equation}\begin{aligned}\notag
    \Xi(\alpha,V,\beta)
    &=\sum_{s}Z(N_s,V,\beta)e^{-\alpha N_s}\\
    &=\sum_{s}\frac{1}{N_s!}Z_1^{N_s}e^{-\alpha N_s}\\
    &=\sum_{s=0}^\infty\frac{1}{N_s!}\left(Z_1e^{-\alpha}\right)^{N_s}\\
    &=\exp\left(Z_1e^{-\alpha}\right)
\end{aligned}\end{equation}
那么
$$\ln\Xi=Z_1e^{-\alpha}$$

Tip:
$$\ln\Xi=Z_1e^{-\alpha}=\overline{N}$$

\paragraph*{能均分定理}\

温度$T$,平衡态,经典系统,系统能量的每一个独立的平方项的平均值等于$\frac{1}{2}kT$。

\tgre{更多的细节来不及写了,,,,,,,,,,实在抱歉!}

\end{document}


\subsection{$\Omega$与玻尔兹曼熵}

一个\textbf{宏观状态}对应的\textbf{微观状态}的个数:(微正则系综为例)
$$\tred{\Omega(N,E,V)}$$

玻尔兹曼熵:
$$S\overset{*}{=}k\ \tred{\ln\Omega}$$

$\tred{\Omega}$不好算,换一个算。先表示出能量低于$E$的所有\textbf{微观状态}的个数之和:
$$\tpur{\Sigma(N,E,V)}=\sum_{E'\leqslant E}\tred{\Omega(N,E',V)}$$
然后再取一厚度为$\Delta$的壳层(对$E$导数乘以厚度):
$$\tgre{\Gamma(N,E,V;\Delta)}\simeq \Delta\frac{\pa \tpur{\Sigma(N,E,V)}}{\pa E}$$
用$\tpur{\Sigma(N,E,V)}$代替$\tred{\Omega(N,E,V)}$。

一般有(跳过推导)
$$\tgre{\ln\Gamma}\simeq\tpur{\ln\Sigma}\simeq\tred{\ln\Omega}$$
那么可以近似表示:
$$S=k\ \tred{\ln\Omega}\simeq k\ \tgre{\ln\Gamma}$$
