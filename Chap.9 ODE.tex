\documentclass[UTF8]{ctexart}
%\usepackage{ctex}
\usepackage{amsmath,amsthm,amsfonts,amssymb,bm}
%\usepackage{xfrac}
\usepackage{graphicx}
\usepackage{float}
\usepackage{geometry}
%\usepackage{subfigure}
%\usepackage{booktabs}
\title{\zihao{1}高等数学 $\cdot$ 常微分方程}
\author{21305337 Betelgeuxe}
\date{仅供学习参考,勿作商业用途}
\geometry{papersize={210mm,297mm}}
\geometry{left=2.9cm,right=2.9cm,top=3.4cm,bottom=2.7cm}
\usepackage{fancyhdr}
\pagestyle{fancy}
\fancyhf{}
\fancyhead[L]{\leftmark\\}
\fancyhead[R]{\rightmark}
\fancyfoot[C]{\thepage}
%\fancyhead[LO]{\chaptername }
\usepackage{hyperref}

\begin{document}
\maketitle
\tableofcontents

\section{基本概念}

\paragraph*{通解}设$y=y(x)$,若n阶常微分方程$F(x,y,y',\cdots,y^{(n)})=0$有解$y=\varphi(x;C_1,\cdots,C_n),$其中$C_1,\cdots,C_n$为独立的$n$个任意常数,则此解为一个通解.

\paragraph*{特解}特解任何一个不包含任意常数的解都称作特解.

\paragraph*{奇解}不包含于通解的解.

\paragraph*{通积分}以隐函数$\varPhi(x,y;C_1,\cdots,C_n)=0$的形式给出的通解.

\paragraph*{上述独立性的判定}
$\displaystyle\frac
{{\rm D}(\varphi,\varphi',\cdots,\varphi^{(n-1)})}
{{\rm D}(C_1,C_2,\cdots,C_n)}
\neq0$,将$\varphi(x;C_1,\cdots,C_n)$看做$C_1,\cdots,C_n$的函数(仅用于雅克比行列式中的求导).

\paragraph*{初值问题}
求方程$F(x,y,y',\cdots,y^{(n)})=0$满足初始条件
$$\begin{cases}
y(x_0)=y_0,\\
y'(x_0)=y_1,\\
\quad \vdots\\
y^{(n-1)}(x_0)=y_{n-1}
\end{cases}$$
的解,其中$x_0,y_0,\cdots,y_{n-1}$是预先给定的常数.

\paragraph*{积分曲线}常微分方程的解的图形称作积分曲线.


\section{初等积分法}

\subsection{变量分离的方程}

\subsubsection{直接积分法(用于变量分离的方程)}

$$
\begin{aligned}
\frac{{\rm d}y}{{\rm d}x}&=f(x)\cdot g(y),\ g(y)\neq0\\
\int\frac{{\rm d}y}{g(y)}&=\int f(x){\rm d}x.
\end{aligned}
$$

\subsubsection{换元法(化为变量分离的方程)}

$$
\begin{aligned}
&1^\circ \ \ y'=f(ax+by+c),\ \underline{\underline{\text{设}z=ax+by+c}}.
\\&2^\circ \ \ y'=f\left(\frac yx\right),\ x\neq0\text{(齐次方程)},\underline{\underline{\text{设}u=\frac yx}},\ \text{则}y'=u+xu'\\
&\ \ \ \implies \frac{{\rm d}u}{{\rm d}x}=\frac{f(u)-u}{x}.\\
&3^\circ \ \ y'=f\left(\frac{a_1x+b_1y+c_1}{a_2x+b_2y+c_2}\right),\\
&\ \ \ 1)\ \ \begin{vmatrix}a_1&b_1\\a_2&b_2\end{vmatrix}\not=0,\text{(求交点后平移换元,转为齐次方程)}
\\&\ \ 联立\begin{cases}a_1x+b_1y+c_1=0\\
a_2x+b_2y+c_2=0\end{cases}\implies(x_0,y_0)\\
&\ \ \ \ \ \ \underline{\underline{设\begin{cases}u=x-x_0,\\v=y-y_0,\end{cases}}}\ \text{则}\frac{{\rm d}u}{{\rm d}v}=f\left(\frac{a_1u+b_1v}{a_2u+b_2v}\right)=h\left(\frac uv\right),\\
&\ \ 2)\ \ \begin{vmatrix}a_1&b_1\\a_2&b_2\end{vmatrix}=0,\text{(分子分母换同一个元,直接达成变量分离)}\hspace{4cm}\\
&\ \ 设\underline{\underline{z=a_1x+b_1y,}}\
\text{则}\frac{{\rm d}z}{{\rm d}x}=a_1+b_1\frac{{\rm d}y}{{\rm d}x}=a_1+b_1f\left(\frac{z+c_1}{kz+c_2}\right).
\end{aligned}
$$

\subsection{一阶线性ODE}

$$
\begin{aligned}
&\frac{{\rm d}y}{{\rm d}x}+P(x)\cdot y=Q(x),\begin{cases}
Q(x)\equiv0,\mbox{齐次线性方程}&(1)\\
Q(x)\not\equiv0,\mbox{非齐次线性方程}&(2)
\end{cases}
\end{aligned}
$$

\subsubsection{一阶线性ODE的一些性质}
$$
\begin{aligned}
&\begin{aligned}
齐次线性方程&\  \cfrac{{\rm d}y}{{\rm d}x}+P(x)\cdot y=
0&(1)\\
非齐次线性方程&\ \cfrac{{\rm d}y}{{\rm d}x}+P(x)\cdot y=Q(x)&(2)
\end{aligned}
\\性质
&①\ \texttt{(1)的任意两个解之和仍为(1)的解}\\
&②\ \texttt{(1)的任一解的常数倍仍为(1)的解}\\
&③\ \texttt{(1)的任一解\textrm{+}\ (2)的任一解\textrm{=}\ (2)的一个解}\\
&④\ \texttt{(2)的任意两个解之差为(1)的解}\\
&⑤\ \texttt{(1)的通解\textrm{+}\ (2)的任一特解\textrm{=}\ (2)的通解.}
\end{aligned}
$$

\subsubsection{一阶线性非齐次ODE}

线性非齐次微分方程$y'+P(x)\cdot y=Q(x)$的解法

第一步:求对应的齐次微分方程(是变量分离的)的通解:
$$y'+P(x)\cdot y=0\implies y=C\cdot y_1(x)$$

第二步:使用\songti\textbf{常数变易法},设$y=u(x)\cdot y_1(x)$将所设$y$代入到原非齐次方程解出$u(x)$(过程中一定会消掉一串式子)

第三步:直接写出通解$y=u(x)\cdot y_1(x)$.


\subsubsection{伯努利方程}

伯努利方程$y'+P(x)\cdot y=Q(x)\cdot y^\alpha\ \ \ (\alpha\neq0,1)$的解法:

以$y^\alpha $同除两边,得到: 
$\left(y^{1-\alpha}\right)'+(1-\alpha)P(x)\cdot y^{1-\alpha}=(1-\alpha)Q(x)$
已经转为线性微分方程.


\subsection{全微分方程}


$$P(x,y){\rm d}x+Q(x,y){\rm d}y=0\ \ \ (1),\ \ \ \exists u(x,y)\
{\rm st.\ }{\rm d}u=P{\rm d}x+Q{\rm d}y$$

则称$(1)$为\songti\textbf{全微分方程}或\songti\textbf{恰当方程}.
注:全微分方程的判定与$\S 8$中的四种判定\textbf{第二型曲线积分与路径无关}的方法可分离变量的微分方程属于全微分方程,其中主要的判定方法为:
$$
\frac{\partial P}{\partial y}\equiv\frac{\partial Q}{\partial x},\ \ \ (x,y)\in D$$

\subsubsection{一些常见的二元函数的全微分}

$$
\begin{aligned}
\frac{y{\rm d}x-x{\rm d}y}{y^2}&={\rm d}\left(\frac xy\right)
\\\frac{{\rm d}x}{y}-\frac{{\rm d}y}{x}&={\rm d}\left(\ln\left|\frac xy\right|\right)
\\\frac{y{\rm d}x-x{\rm d}y}{x^2+y^2}&={\rm d}\left(\arctan\frac xy\right)
\\\frac{y{\rm d}x-x{\rm d}y}{x^2-y^2}&=\frac 12{\rm d}\left(\ln\left|\frac {x-y}{x+y}\right|\right)
\end{aligned}
$$

\subsection{积分因子}

若微分方程$$M(x,y){\rm d}x+N(x,y){\rm d}y=0\ \ \ (1)$$不是全微分方程,$\exists \mu(x,y),{\rm st.\ }$ $$\mu M{\rm d}x+\mu N{\rm d}y=0\ \ \ (2) $$ 是全微分方程,则称$\mu(x,y)$为$(1)$的\songti\textbf{积分因子}.
积分因子必须满足:
$$
\frac{\partial (\mu M)}{\partial y}\equiv\frac{\partial (\mu N)}{\partial x}$$

\subsubsection{“一元积分因子”}

\paragraph*{$X$的“一元积分因子”}若$$F=\frac{\displaystyle\frac{\partial M}{\partial y}-\frac{\partial N}{\partial x}}{N}$$是$x$的一元函数$F(x)$,则$$\mu(x)=e^{\int_{x_0}^xF(t){\rm d}t}$$是一个积分因子.

\paragraph*{$Y$的“一元积分因子”}类似地,若$$G=\frac{\displaystyle\frac{\partial M}{\partial y}-\frac{\partial N}{\partial x}}{M}$$是$y$的一元函数$G(y)$,则$$\mu(y)=e^{\int_{y_0}^yG(t){\rm d}t}$$是一个积分因子.

\subsubsection{齐次方程的积分因子}

对于齐次方程
$$\begin{aligned}
&P(x,y){\rm d}x+Q(x,y){\rm d}y=0\\
\texttt{即}\ \ &y'=-\frac{P(x,y)}{Q(x,y)}=f\left(\frac yx\right)  \end{aligned}$$

则此齐次方程的一个积分因子为:
$$
\mu=\frac{1}{xP+yQ}
$$

\subsection{可降阶的二阶ODE}

\subsubsection{$y''=f(x,y')$型二阶ODE}

设$y'=P(x)$,则$y''=P'$,原方程化为一阶方程$P'=f(x,P)$,设其通解为$P=\varphi (x,C_1)$,两边对$x$积分得到原方程的通解$y=\int \varphi(x,C1){\rm d}x.$

\subsubsection{$y''=f(y,y')$型二阶ODE}

设$y'=P(y)$,则$y''=P\cdot P'$,原方程化为一阶方程$P\cdot P'=f(y,P)$,设其通解为$P=\psi (y,C_1)$,分离变量后积分得到原方程的通解$\int {\rm d}y/\psi(y,C1)=x.$

\section{ODE解的存在唯一性定理}

若初值问题$\begin{cases}y'=f(x,y),\\y(x_0)=y_0\end{cases}$中的$f(x,y)$在闭矩形域$$R=\{(x,y)\big|\ |x-x_0|\leq a,|y-y_0|\leq b\}$$连续,且在此矩形域内对$y$满足\songti\textbf{李普希兹条件},即$$|f(x,y_1)-f(x,y_2)|\leq L|y_1-y_2|,\ \ (x,y_1),(x,y_2)\in R$$则此初值问题在区间$[x_0-h,x_0+h]$上有且只有一个解,其中常数$h=\min(a,\frac{b}{M}),\ \ M=\max\{|f(x,y)|,(x,y)\in R\}$.


\subsection{皮卡序列的作法}


一次近似解:$\displaystyle{y_1(x)}={y_0}+\int_{x_0}^x f(x,{y_0}){\rm d}x,\ \ x\in[x_0-h,x_0+h]$,

二次近似解:$\displaystyle{y_2(x)}={y_0}+\int_{x_0}^x f\big(x,{y_1(x)}\big){\rm d}x,\ \ x\in[x_0-h,x_0+h]$,

三次近似解:$\displaystyle{y_3(x)}={y_0}+\int_{x_0}^x f\big(x,{y_2(x)}\big){\rm d}x,\ \ x\in[x_0-h,x_0+h]$,

$\ \ \ \ \ \vdots \ \ \ \ \ \ \ \ \ \ \ \ \ \ \ \ \ \ \ \ \ \ \ \ \ \ \ \ \ \ \ \ \ \ \ \ \ \ \ \ \ \vdots$

\section{高阶线性ODE}
形如
$$y^{n}(X)+p_{n-1}(x)y^{(n-1)}+\cdots+p_1(x)y'+p_0(x)y=f(x)$$
的方程称为$n$阶线性微分方程,其中$p_i(x)$连续.

\paragraph*{定理}各阶线性微分方程均存在唯一性定理.

\paragraph*{定义}类似线性代数中的定义,可用线性组合定义函数组$\varphi_1(x),\varphi_2(x),\cdots,\varphi_n(x)$在区间$[a,b]$上的\songti\textbf{线性无关}与\songti\textbf{线性相关}.

\subsection{线性方程解的叠加原理}
类似一阶情形,高阶线性非齐次方程的解与其对应的齐次方程的解有一系列关系.

其中较重要的有:
$$m\cdot\texttt{齐特解}_1+n\cdot\texttt{齐特解}_2=\texttt{齐特解}_3$$
$$\texttt{齐通解}+\texttt{非齐特解}=\texttt{非齐通解}\quad\boxed{Y+y^\ast=y}$$
$$\texttt{线性齐次方程中},n\texttt{个线性无关的特解的线性组合是其通解}$$

若$a_{n-1},\cdots,a_0$为常数,且$$\begin{cases}
    y^{(n)}+a_{n-1}y^{(n-1)}+\cdots+a_1y'+a_0y=f_1(x)\text{的特解为}y_1^\ast,\\
    \qquad\vdots\\ 
    y^{(n)}+a_{n-1}y^{(n-1)}+\cdots+a_1y'+a_0y=f_1(x)\text{的特解为}y_t^\ast,
\end{cases}$$

则$y^{(n)}+a_{n-1}y^{(n-1)}+\cdots+a_1y'+a_0y=f_1(x)+\cdots+f_t(x)$的特解为$y=y_1^\ast+\cdots+y_t^\ast$.

\subsection{朗斯基(Wronski)行列式}
对于函数组$\varphi_1(x),\varphi_2(x),\cdots,\varphi_n(x)$,朗斯基行列式
$$W(x)=\begin{vmatrix}
    \varphi_1(x)&\varphi_2(x)&\cdots&\varphi_n(x)\\
    \varphi_1'(x)&\varphi_2'(x)&\cdots&\varphi_n'(x)\\
    \vdots      &\vdots     &\ddots &\vdots\\
    \varphi_1^{(n-1)}(x)&\varphi_2^{(n-1)}(x)&\cdots&\varphi_n^{(n-1)}(x)\\
\end{vmatrix}$$
$$W(x)\equiv 0\Longleftrightarrow \varphi_1(x),\varphi_2(x),\cdots,\varphi_n(x)\texttt{线性相关}.$$

可以证明,朗斯基行列式要么恒为零,要么无零点.

\noindent 注:雅克比行列式判定通积分中多个任意常数是否独立,朗斯基行列式判定函数组是否线性无关,朗斯基行列式是雅克比行列式在通解为函数组线性组合的情形.

\subsection{线性常系数齐次ODE与特征方程}
对微分方程$$y^{(n)}+a_{n-1}y^{(n-1)}+\cdots+a_1y'+a_0y=0,\ $$其中$a_{n-1},\cdots,a_0$为常数,\songti\textbf{特征方程}为:
$$\lambda^{n}+a_{n-1}\lambda^{n-1}+\cdots+a_1\lambda+a_0=0$$
可解得$n$个特征根为$\lambda_1,\lambda_2,\cdots,\lambda_n$,根据下表,可写出每个特征根所对应的线性无关的特解(或$k$个特征根对应$k$个特解),从而表示出上述微分方程的通解.
\begin{table}[H]
    \centering
    \renewcommand{\arraystretch}{1.13}
    \begin{tabular}{l|l}
    \hline
    特征根&对应的线性无关的特解\\
    \hline
    单实根$\lambda$ &$e^{\lambda x}$\\
    $k$重实根$\lambda(k>1)$ &$e^{\lambda x},xe^{\lambda x},\cdots,x^{k-1}e^{\lambda x}$\\
    单共轭复根$\lambda_{1,2}=\alpha\pm i\beta$ &$e^{\alpha x}\cos\beta x,e^{\alpha x}\sin\beta x$\\
    $k$重共轭复根$\lambda_{1,2}=\alpha\pm i\beta(m>1)$ &$e^{\alpha x}\cos\beta x,e^{\alpha x}\sin\beta x,\ xe^{\alpha x}\cos\beta x,xe^{\alpha x}\sin\beta x$\\
    &\qquad\qquad $\cdots,x^{m-1}e^{\alpha x}\cos\beta x,x^{m-1}e^{\alpha x}\sin\beta x$\\
    \hline
    \end{tabular}
\end{table} 

\subsection{若干特殊线性常系数非齐次ODE的特解(待定系数法)}
高阶微分方程$y^{(m)}+b_{m-1}y^{(m-1)}+\cdots+b_1y'+b_0y=f(x)$中,若$f(x)$的形式在下表列出,则特解可以直接根据下表设出.(待定系数)

\begin{table}[H]
    \centering
    \renewcommand{\arraystretch}{1.3}
    \begin{tabular}{c|c|c}
    \hline
    $f(x)$的形式&条件&特解的形式\\
    \hline
    $\boxed{P_n(x)}\cdot \boxed{e^{\alpha x}}$&$\alpha\not=\lambda$&$Q_n(x)e^{\alpha x}$\\
    &$\alpha=\lambda(k\text{重})$&$x^kQ_n(x)e^{\alpha x}$\\
    \hline
    $\boxed{P_n(x)}\cdot \boxed{e^{\alpha x}}\cdot \boxed{(a\cos\beta x+b\sin\beta x)}$&$\alpha\pm {\rm i}\beta\neq\lambda$&$[Q_n(x)\cos\beta x+R_n(x)\sin\beta x]e^{\alpha x}$\\
    $(\beta\neq 0)$&$\alpha\pm {\rm i}\beta=\lambda(k\text{重})$&$x^k[Q_n(x)\cos\beta x+R_n(x)\sin\beta x]e^{\alpha x}$\\
    \hline
    \end{tabular}
\end{table} 
再对待定系数的特解求$m$次导数,与原ODE对比系数,可以将待定的系数全部解出.

\noindent 注:(1)$P_n(x),Q_n(x),R_n(x)$为$n$次多项式,$Q_n(x),R_n(x)$各需要待定$n+1$个系数.

(2)$\alpha(\pm{\rm i}\beta)=k$重$\lambda$时,特解代入到微分方程对比系数时,只有$0\sim n$次项的验证有用,最高的若干项可能不必验证.

\subsection{二阶线性非齐次ODE与常数变易法($f(x)$符合特定形式时可用待定系数法代替)}
\noindent 设已求出$$y''+p(x)\cdot y'+q(x)\cdot y=0$$的两个线性无关的特解$\varphi_1(x),\varphi_2(x)$,则可设$$y''+p(x)\cdot y'+q(x)\cdot y=f(x)$$的特解为:$$y=u_1(x)\cdot \varphi_1(x)+u_2(x)\cdot \varphi_2(x)$$

在约束条件$$u_1'\varphi_1+u_2'\cdot\varphi_2=0$$下,求出$y',y''$(求导可跳过,直接记住以下方程组即可)后代入到原ODE可得:
$$\begin{cases}
    u_1'\varphi_1+u_2'\varphi_2=0\\
    u_1'\varphi_1'+u_2'\varphi_2'=f(x)
\end{cases}$$
直接解出$u_1',u_2'$,再直接积分求得$u_1,u_2.$

\subsection{欧拉方程}
$$a_nx^ny^{(n)}+a_{n-1}x^{n-1}y^{(n-1)}+\cdots+a_1xy'+a_0y=0$$
其中$a_i$为常数.

$x>0$时设$x=e^t$,再将$y$对$x$的各阶导数转换为$y$对$t$的导数后,代入到原ODE,可消去所有$x=e^t$项,得到常系数线性ODE.




$$
\begin{aligned}
&
\end{aligned}
$$

\end{document}